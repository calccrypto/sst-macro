% !TEX root = developer.tex

\chapter{\sstmacro Classes}\label{chapter:classes}

\section{Class Style}\label{sec:style}

\subsection{Basic Class}\label{subsec:basicClass}
Most classes are manually managed, being explicitly deleted.  Whenever possible,
smart pointers should be avoided since they create thread-safety headaches.
For cases where manual deletion is cumbersome, classes in \sstmacro can inherit from the top-level class \inlinecode{ptr_type}, 
which exists in \inlinecode{namespace sprockit}. 
The examples here can be found in the code repository in \inlineshell{tutorials/programming/basic}.

To begin, you just need to include the appropriate header file, declare the namespace desired, and start the class declaration (see \inlineshell{illustration.h}).

\begin{CppCode}
#include <sprockit/ptr_type.h>

namespace sstmac {
namespace tutorial {

class illustration :
  public sprockit::ptr_type
{
\end{CppCode}
To simplify reading of the code (i.e. not typing \inlinecode{sprockit::refcount_ptr} everywhere), every class is required to typedef itself.

\begin{CppCode}
 public:
  typedef sprockit::refcount_ptr<illustration> ptr;
  typedef sprockit::refcount_ptr<const illustration> const_ptr;
\end{CppCode}
The reference-counted pointer type can now be used as \inlinecode{illustration::ptr}.

After declaring public typedefs, the public function interface can be declared.
Every \inlinecode{ptr_type} class that is instantiated must implement a \inlinecode{to_string} function,
mostly used for debugging purposes.

\begin{CppCode}
 public:
  std::string
  to_string() const override {
    return "message class";
   }
\end{CppCode}

Public set/get functions can be added for member variables, if desired.
We have generally followed snake\_case, using lower-case letters and underscores.

\begin{CppCode}
  std::string
  message() const {
    return message_;
   }

  void
  set_message(const std::string& msg){
    message_ = msg;
  }
\end{CppCode}
Unfortunately, amongst the various developers, a uniform style was never agreed.
Some accessor functions could be called \inlinecode{get_message}, but most avoid the prefix and would be called just \inlinecode{message}.
After any \inlinecode{private}/\inlinecode{protected} functions, member variables can be declared.
\sstmacro developers have not historically been very strict with use of \inlinecode{protected} vs. \inlinecode{private} for members, typically using \inlinecode{protected} as a default for convenience. 

NOTE: We already have our first style violation here.  The entire class is implemented here in a header file.
In general, unless a trivial function like set/get, put the function implementation into a .cc file.

\subsection{Inheritance and Parent Classes}\label{subsec:inheritance}
We can also work through an example with inheritance.  
If you are going to use smart pointers, inheritance is somewhat tricky.  
Here we demonstrate in detail how to use smart pointers with virtual inheritance, 
which is a scenario that often occurs with the \inlineshell{sst_message} class.
Many classes will never actually be instantiated in \sstmacro - they are virtual classes defining an abstract interface.
In the file \inlineshell{gem.h}, we define the \inlinecode{gem} abstract interface. Again, we include the header,
declare namespaces, and declare the class.

\begin{CppCode}
#include <sprockit/ptr_type.h>

namespace sstmac {
    namespace tutorial {
    
class gem :
  virtual public sprockit::ptr_type
{
\end{CppCode}
We note here the use of virtual inheritance, which will become important later.
In general, you should almost always use virtual inheritance from \inlinecode{ptr_type}.
Even though this class is not a complete type (pure virtual functions), we still need the public pointer typedefs.

\begin{CppCode}
 public:
  typedef sprockit::refcount_ptr<gem> ptr;
  typedef sprockit::refcount_ptr<const gem> const_ptr;
\end{CppCode}
We then define the abstract \inlinecode{gem} interface

\begin{CppCode}
 public:
  virtual int
  value() const = 0;

  virtual ~gem(){}
\end{CppCode}
There is no static \inlinecode{construct} function!
The static construct function
is a signal that the class is a \emph{complete-type}, with all virtual methods implemented.
Because of C++ destructors for abstract classes, you \emph{must} define a virtual destructor for every parent class - even if it does nothing.
Without this, the memory management system will not work correctly.
Many tutorials on the subtleties of virtual destructors can be found online.

We can repeat the same process for a new abstract interface called \inlinecode{mineral}.
Declare the class:

\begin{CppCode}
#include <sprockit/ptr_type.h>

namespace sstmac {
    namespace tutorial {

class mineral :
    virtual public sprockit::ptr_type
{
\end{CppCode}
Again, we use virtual inheritance. We declare public typedefs:

\begin{CppCode}
 public:
  typedef sprockit::refcount_ptr<mineral> ptr;
  typedef sprockit::refcount_ptr<const mineral> const_ptr;
\end{CppCode}
We then define the abstract \inlinecode{mineral} interface:

\begin{CppCode}
 public:
  virtual std::string
  structure() const = 0;

  virtual ~mineral(){}
\end{CppCode}

Now we want to create a specific instance of a \inlinecode{gem} and \inlinecode{mineral}.

\begin{CppCode}
#include "gem.h"
#include "mineral.h"

namespace sstmac {
    namespace tutorial {

class diamond :
    public gem,
    public mineral
{
\end{CppCode}
The reason for using virtual inheritance is now evident.
The \inlinecode{diamond} class is going to implement both the \inlinecode{gem} and \inlinecode{mineral} interfaces.
However, both classes separately inherit from \inlinecode{ptr_type}.
We thus have the multiple inheritance diamond problem. 
For this reason, both parent classes must virtually inherit from \inlinecode{ptr_type}.
Most of this discussion is just a basic C++ tutorial without really being \sstmacro specific.
However, this diamond problem is so ubiquitous with \inlinecode{ptr_type}, it merits a discussion here in the programmer's reference.

We can now fill out the class.  First, we have the public typedefs

\begin{CppCode}
 public:
  typedef sprockit::refcount_ptr<diamond> ptr;
  typedef sprockit::refcount_ptr<const diamond> const_ptr;
\end{CppCode}


The \inlinecode{ptr_type} class requires we add a \inlinecode{to_string} function

\begin{CppCode}
std::string
to_string() const override {
  return "diamond";
}
\end{CppCode}

We now complete the gem interface

\begin{CppCode}
int
value() const {
  return num_carats_ * 100;
}
\end{CppCode}

and the mineral interface
\begin{CppCode}
std::string
structure() const {
  return "tetrahedral carbon";
}
\end{CppCode}

And the constructor

\begin{CppCode}
  diamond(num_carats)
   : num_carats_(num_carats)
{
}
\end{CppCode}

Finally, we declare member variables

\begin{CppCode}
 protected:
  int num_carats_;
\end{CppCode}


\subsection{Exceptions}
\label{classes:style:basic:exceptions}

The \inlinecode{ptr_type.h} header file automatically includes a common set of exceptions, all in \inlinecode{namespace sprockit}.
The most notable are:
\begin{itemize}
\item \inlinecode{value_error}: A parameter value is wrong
\item \inlinecode{unimplemented_error}: A place-holder to indicate this function is not valid or not yet done
\item \inlinecode{illformed_error}: A catch-all for anything that fails a sanity check
\item \inlinecode{null_error}: A null object was received
\end{itemize}
\sprockit provides a macro for throwing exceptions.
It should always be used since they provide extra metadata to the exception like file and line number.
The macro, \inlinecode{spkt_throw_printf}, takes two mandatory arguments: the exception type and an exception message.
An arbitrary number of arguments can be given, that get passed a \inlinecode{printf} invocation, e.g.

\begin{CppCode}
spkt_throw_printf(value_error, "invalid number of carats %d", num_carats_);
\end{CppCode}
 

\section{Factory Types}\label{sec:factory}
We here introduce factory types, i.e. polymorphic types linked to keywords in the input file.
String parameters are linked to a lookup table, finding a factory that produces the desired type.
In this way, the user can swap in and out C++ classes using just the input file.
There are many distinct factory types relating to the different hardware components.
There are factories for topology, NIC, node, memory, switch, routing algorithm - the list goes on.
Here show how to declare a new factory type and implement various polymorphic instances.
The example files can be found in \inlineshell{tutorials/programming/factories}.

\subsection{Usage}\label{subsec:usage}
Before looking at how to implement factory types, let's look at how they are used.
Here we consider the example of an abstract interface called \inlinecode{actor}.
The code example is found in \inlineshell{main.cc}. The file begins

\begin{CppCode}
#include <sstmac/skeleton.h>
#include "actor.h"

namespace sstmac {
    namespace tutorial {

#define sstmac_app_name rob_reiner

int
main(int argc, char **argv)
{
\end{CppCode}
The details of declaring and using external apps is found in the user's manual.
From here it should be apparent that we defined a new application with name \inlinecode{rob_reiner}
which is invoked via the \inlinecode{main} function.
Inside the main function, we create an object of type \inlinecode{actor}.

\begin{CppCode}
actor* the_guy = actor_factory::get_param("actor_name", get_params());
the_guy->act();
return 0;
\end{CppCode}
We use the \inlinecode{actor_factory} to create the object.
The value of \inlineshell{actor_name} is read from the input file \inlineshell{parameters.ini} in the directory.
Depending on the value in the input file, a different type will be created.
The input file contains several parameters related to constructing a machine model - ignore these for now.
The important parameters are:

\begin{ViFile}
app1.name = rob_reiner
biggest_fan = jeremy_wilke
actor_name = patinkin
sword_hand = right
\end{ViFile}

Using the Makefile in the directory, if we compile and run the resulting executable we get the output

\begin{ViFile}
Hello. My name is Inigo Montoya. You killed my father. Prepare to die!
Estimated total runtime of           0.00000000 seconds
SST/macro ran for       0.0025 seconds
\end{ViFile}

If we change the parameters:

\begin{ViFile}
app1.name = rob_reiner
biggest_fan = jeremy_wilke
actor_name = guest
num_fingers = 6
\end{ViFile}

we now get the output

\begin{ViFile}
You've been chasing me your entire life only to fail now.
I think that's the worst thing I've ever heard. How marvelous.
Estimated total runtime of           0.00000000 seconds
SST/macro ran for       0.0025 seconds
\end{ViFile}

Changing the values produces a different class type and different behavior.
Thus we can manage polymorphic types by changing the input file.

\subsection{Base Class}\label{subsec:baseClass}
To declare a new factory type, you must include the factory header file

\begin{CppCode}
#include <sprockit/factories/factory.h>

namespace sstmac {
    namespace tutorial {

class actor
{
\end{CppCode}


We now define the public interface for the actor class

\begin{CppCode}
 public:
  actor(sprockit::sim_parameters* params);
  
  virtual void
  act() = 0;

  virtual ~actor(){}
\end{CppCode}
Again, we must have a public, virtual destructor.
Each instance of the actor class must implement the \inlinecode{act} method.

For factory types, each class must take a parameter object in the constructor.
The parent class has a single member variable

\begin{CppCode}
 protected:
  std::string biggest_fan_;
\end{CppCode}

After finishing the class, we need to invoke a macro

\begin{CppCode}
DeclareFactory(actor);
\end{CppCode}
making \sstmacro aware of the new factory type.

Moving to the \inlineshell{actor.cc} file, we see the implementation

\begin{CppCode}
namespace sstmac {
    namespace tutorial {

actor::actor(sprockit::sim_parameters* params)
{
  biggest_fan_ = params->get_param("biggest_fan");
}
\end{CppCode}
We initialize the member variable from the parameter object.  We additionally need a macro

\begin{CppCode}
ImplementFactory(sstmac::tutorial::actor);
\end{CppCode}
that defines certain symbols needed for implementing the new factory type.
For subtle reasons, this must be done in the global namespace.

\subsection{Child Class}\label{subsec:childClass}
Let's now look at a fully implemented, complete actor type.  We declare it

\begin{CppCode}
#include "actor.h"

namespace sstmac {
    namespace tutorial {

class mandy_patinkin :
    public actor
{
 public:
  mandy_patinkin(sprockit::sim_parameters* params);
\end{CppCode}

We have a single member variable

\begin{CppCode}
 private:
  std::string sword_hand_;
\end{CppCode}

This is a complete type that can be instantiated. 
To create the class we will need the constructor:

\begin{CppCode}
mandy_patinkin(sprockit::sim_parameters* params);
\end{CppCode}

And finally, to satisfy the \inlinecode{actor} public interface, we need

\begin{CppCode}
virtual void
act() override;
\end{CppCode}

Moving to the implementation, we must first register the new type using the macro

\begin{CppCode}
namespace sstmac {
    namespace tutorial {

SpktRegister("patinkin", actor, mandy_patinkin,
    "He's on one of those shows now... NCIS? CSI?");
\end{CppCode}
The first argument is the string descriptor that will be linked to the type.
The second argument is the parent, base class. 
The third argument is the specific child type.
Finally, a documentation string should be given with a brief description.
Whatever string value is registered here will be used in the input file to create the type.
We can now implement the constructor:

\begin{CppCode}
mandy_patinkin::mandy_patinkin(sprockit::sim_parameters* params) :
  actor(params)
{
  sword_hand_ = params->get_param("sword_hand");

  if (sword_hand_ == "left"){
    spkt_throw(value_error, "I am not left handed!");
  }
  else if (sword_hand_ != "right"){
      spkt_abort_printf(value_error,
          "Invalid hand specified: %s",
          sword_hand_.c_str());
  }
}
\end{CppCode}
The child class must invoke the parent class method. 
Finally, we specify the acting behavior

\begin{CppCode}
void
mandy_patinkin::act()
{
    std::cout << "Hello. My name is Inigo Montoya. You killed my father. Prepare to die!"
              << std::endl;
}
\end{CppCode}

Another example \inlineshell{guest.h} and \inlineshell{guest.cc} in the code folder shows the implementation for the second class.

\subsection{External Linkage}\label{subsec:linkage}
If you glance at the Makefile, you will see how and why the executable is created.
A compiler wrapper \inlineshell{sst++} points the makefile to the \sstmacro libraries, including a library \inlineshell{libsstmac_main} that actually implements the \inlinecode{main} routine and the \sstmacro driver.
By creating another executable, arbitrary code can be linked together with the core \sstmacro framework.
The same \sstmacro driver is invoked in the external executable as would be invoked by the default main executable.

